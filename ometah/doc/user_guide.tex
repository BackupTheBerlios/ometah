%% LyX 1.3 created this file.  For more info, see http://www.lyx.org/.
%% Do not edit unless you really know what you are doing.
\documentclass[english]{article}
\usepackage[T1]{fontenc}
\usepackage[latin1]{inputenc}
\usepackage{float}
\usepackage{graphicx}

\makeatletter

%%%%%%%%%%%%%%%%%%%%%%%%%%%%%% LyX specific LaTeX commands.
\floatstyle{ruled}
\newfloat{algorithm}{tbp}{loa}
\floatname{algorithm}{Algorithm}

%%%%%%%%%%%%%%%%%%%%%%%%%%%%%% Textclass specific LaTeX commands.
 \newenvironment{lyxcode}
   {\begin{list}{}{
     \setlength{\rightmargin}{\leftmargin}
     \setlength{\listparindent}{0pt}% needed for AMS classes
     \raggedright
     \setlength{\itemsep}{0pt}
     \setlength{\parsep}{0pt}
     \normalfont\ttfamily}%
    \item[]}
   {\end{list}}

%%%%%%%%%%%%%%%%%%%%%%%%%%%%%% User specified LaTeX commands.
\usepackage{babel}

\usepackage{babel}
\makeatother
\begin{document}

\title{User guide}

\maketitle

\section{Open Metaheuristic modules}

The Open metaheuristics library is separated in several modules:

\begin{description}
\item [ometah:]the core of the library and the main command line interface,
\item [ometahlab:]a set of scripts for algorithms tests and graphical outputs,
\item [registration:]a example of application (rigid registration of images).
\end{description}

\section{Main user interface}

The current command line options can be listed by typing:

\begin{lyxcode}
ometah~-{}-help
\end{lyxcode}

\subsection{Basic features}

todo


\subsection{Advanced features}

todo


\section{Code structure}


\subsection{Framework}


\subsubsection{Adaptive Learning Search}

In oMetah, metaheuristics are considered as adaptive learning searches
(ALS). The basic steps of such iterative algorithms are shown on the
algorithm~\ref{alg:ALS_base}.

%
\begin{algorithm}

\caption{\label{alg:ALS_base}Adaptive Learning Search basic algorithm.}

\begin{description}
\item [Initialize]a sample (a set of points);
\item [Iterate]until stopping criteria:

\begin{description}
\item [Using]a sample of the objective function;
\item [Learning:]the algorithm extracts {}``interesting'' informations
from the sample;
\item [Diversification:]it searches for new solutions;
\item [Intensification:]it searches to improve the existing sample;
\item [Replace]the previous sample with the new one.
\end{description}
\item [End]~
\end{description}

\end{algorithm}



\subsubsection{Components}

The framework is composed of three main component: \emph{metaheuristics},
\emph{problems} and \emph{communication}. In order to facilitate an
easy connection with third party softwares, metaheuristics and problems
are separated. Metaheuristics are asking problems the value(s) of
a point through a communication interface. The communication component
follows a client/server design, permiting to use several communication
protocols. 

%
\begin{figure}
\begin{center}\includegraphics[%
  scale=0.4]{ometah_communication.pdf}\end{center}


\caption{\label{cap:UML-communication}UML diagram of relationships between
metaheuristics and problems in \emph{oMetah}.}
\end{figure}



\subsubsection{Classes}

The structure of the framework follows two design patterns: the \emph{template}
and the \emph{abstract factory}. The abstract factory is only used
in the main interface, to facilitate the manipulation of several instances
of different classes into a single container. The template pattern
is used in the three main component of the framework.


\paragraph{Creating classes with the template}

The Template Method is a quite simple behavioral pattern, it only
consists in defining the skeleton of an algorithm, while allowing
subclasses to define particular steps of the algorithm. It can be
used for the metaheuristics classes (see figure \ref{cap:UML-template}),
which need to share basic attributes or methods (sample, output, etc.)
but will implement differently common methods (diversification, intensification
and learning) that will be called by the main procedure. The same
structure is used for both problems and communication interfaces.

%
\begin{figure}
\begin{center}\includegraphics[%
  scale=0.4]{als_template.pdf}\end{center}


\caption{\label{cap:UML-template}UML diagram of the template pattern, used
to implement a metaheuristic as an \emph{ALS}.}
\end{figure}



\paragraph{Manipulating instances with the abstract factory and the Set}

Abstract Factory belongs to the class of creation design patterns.
It provides an easy way to create objects sharing common properties
but one won't have to explicitly specify which classes. In a simple
way, one will define one abstract factory, as an instance of an abstract
class, that will be used to create instances of derived classes, via
their own factories.

For example, an abstract factory would allow to create instances of
specific metaheuristics, so as to, for example, add them to a specific
set, by using the same interface for all. One only has to define a
subclass of the metaheuristic factory for each metaheuristic that
we program (see figure \ref{cap:UML-abstract-factory}). As a result,
one has not to specify the type of objects wanted, it even remains
unknown. 

%
\begin{figure}
\begin{center}\includegraphics[%
  scale=0.4]{als_manipulating.pdf}\end{center}


\caption{\label{cap:UML-abstract-factory}UML diagram of the abstract factory
pattern, used to manipulate metaheuristics implementations.}
\end{figure}


%
\begin{figure}
\begin{center}\includegraphics[%
  scale=0.4]{ometah_set.pdf}\end{center}


\caption{\label{cap:UML-set-class}UML diagram of the set class.}
\end{figure}


In order to handle the different object types, we use an abstraction
of a set of objects (see figure~\ref{cap:UML-set-class}). Thanks
to this class and to the abstract factory pattern, it is possible
to manipulate several objects sharing a common interface.This class
can be used to connect several metaheuristics with several problems
through several communication protocols.


\subsection{Test toolbox}

todo


\section{Howto}


\subsection{Using}


\subsubsection{Simple Optimization}

Optimizing the Sphere problem with the HCIAC algorithm, with a maximum
of 100 evaluations:

\begin{lyxcode}
ometah~-p~Sphere~-m~HCIAC~-e~100
\end{lyxcode}
Optimizing the Rosenbrock problem on 3 dimensions with the CHEDA algorithm,
with 100 evaluation over 10 iterations for a sample of 10 points:

\begin{lyxcode}
ometah~-p~Rosenbrock~-d~3~-m~CHEDA~-e~100~-i~10~-s~10
\end{lyxcode}

\subsubsection{Testing a metaheuristic on one problem}

todo


\subsubsection{Testing a metaheuristic on a problem set}

todo


\subsubsection{Comparing several metaheuristic}

todo


\subsubsection{Advanced comparisons and tests}

todo


\subsection{Programming}


\subsubsection{Using the framework}

In order to build your own program with the oMetah framework, you
can either link it with the dynamic library or compile it staticaly.
In the first case you can use the license you choose, but note that
in the second case, your program should be released under the LGPL
terms.

To link your own work with library, you should install the headers
in the library path of your system (for example \texttt{/usr/local/lib}
or \texttt{/usr/lib} on a POSIX system).

You should include the headers you want to use, with the following
instruction: \texttt{\#include <ometah/SUBDIRECTORY/HEADER.hpp>},
and then compile your program, specifying to the compiler that you
want to link it dynamically with oMetah.


\subsubsection{Building the library}

Open Metaheuristic use scons as a construction tool, to build the
default targets, simply type \texttt{scons}. All the instructions
on how to build the library and the official interface are located
in the \texttt{SConstruct} file. To get some help on what can be built,
simply try \texttt{scons -h}.

Basically, you can build the official interface and the library, either
statically or dynamically, and with or without debug informations.
Some examples:

\begin{itemize}
\item \texttt{scons ometah static=yes debug=no}: build a static version
of the main interface, without debugging informations,
\item \texttt{scons ometah static=0}: build a the shared library and the
interface dynamically linked,
\item \texttt{scons -c ometah}: clean the files used to generate the interface,
\item \texttt{scons -c install}: uninstall the files,
\item \texttt{scons -c}: clean all the files used for the default build.
\end{itemize}

\subsubsection{Create a user interface}

You can use the library through the official interface (defined in
\texttt{interface/ometah.cpp}), but you can also build your own interface.
An simple example of how to procede is given in \texttt{examples/example\_interface.cpp}.

Firstly, you need to include the headers you want to use: here we
choose an estimation of distribution algorithm as the metaheuristic,
some problems from a benchmark and the embedded client/server communication.

\begin{lyxcode}
\#include~<ometah/metaheuristic/estimation/itsEstimationOfDistribution.hpp>

\#include~<ometah/problem/CEC05/itsCEC05\_SSRPO\_Base.hpp>

\#include~<ometah/communication/itsCommunicationServer\_embedded.hpp>

\#include~<ometah/communication/itsCommunicationClient\_embedded.hpp>
\end{lyxcode}
The next step consist in instanciating the objects you will use: the
metaheuristic, the problem and the communication client/server:

\begin{lyxcode}
itsEstimationOfDistribution~metaheuristic;

itsAckley~problem;

itsCommunicationClient\_embedded~client;

itsCommunicationServer\_embedded~server;
\end{lyxcode}
In order to use the communication client/server approach, you will
need to specify some options. When using the embedded way, you just
need to specify pointers to objects:

\begin{lyxcode}
metaheuristic.problem~=~\&~client;

client.problem~=~\&~server;

server.problem~=~\&~problem;
\end{lyxcode}
After all, you just need to start the optimization:

\begin{lyxcode}
metaheuristic.start();
\end{lyxcode}

\subsubsection{Add a new metaheuristic}

Adding a metaheuristic consist in using the \texttt{itsMetaheuristic}
class to define an ALS interface, and implement it.

\begin{lyxcode}
class~ITSEXAMPLEMETAHEURISTIC~:~public~itsMetaheuristic~

\{~

protected:

 void~learning();

 void~diversification();

 void~intensification();

public:

 ITSEXAMPLEMETAHEURISTIC();

 \textasciitilde{}ITSEXAMPLEMETAHEURISTIC();

\};
\end{lyxcode}
There is two important method to implement: the constructor, where
informations on the method are inserted, and one of the trhee steps
of the ALS (learning, diversification, intensification). The best
way is to separate your algorithm in the three steps, but if only
one is suitable, you can define empty functions.


\subsubsection{Add a new problem}

Adding a problem consist in using the \texttt{itsProblem} class to
define an common interface:

\begin{lyxcode}
class~ITSEXAMPLEPROBLEM~:~public~itsProblem~

\{

public:

~ITSEXAMPLEPROBLEM();

~itsPoint~objectiveFunction(itsPoint~point);~

\};
\end{lyxcode}
The two important method here are the constructor, where characteristics
of the problem are inserted, and the \texttt{objectiveFunction} method,
where the concrete problem is implemented.


\subsubsection{Add a communication interface}

todo
\end{document}
